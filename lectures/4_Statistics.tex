\documentclass[aspectratio=169,11pt]{beamer}

\usetheme{Singapore}
\usepackage[utf8]{inputenc}
\usepackage{amsmath}
\usepackage{amsfonts}
\usepackage{amssymb}
\usepackage{graphicx}
\usepackage{hyperref}

% Setup the bibliography
\usepackage[style=authortitle,backend=bibtex]{biblatex}
\addbibresource{bibliography.bib}
\setbeamertemplate{bibliography item}[text]
\setbeamerfont{footnote}{size=\tiny}

% Allow footnotes with no number
\newcommand\blfootnote[1]{%
  \begingroup
  \renewcommand\thefootnote{}\footnote{#1}%
  \addtocounter{footnote}{-1}%
  \endgroup
}

% Allow section title slides
\AtBeginSection[]{
  \begin{frame}
  \vfill
  \centering
  \begin{beamercolorbox}[sep=8pt,center,shadow=true,rounded=true]{title}
    \usebeamerfont{title}\insertsectionhead\par%
  \end{beamercolorbox}
  \vfill
  \end{frame}
}

\author{Dr Stephen Pederson}
\title{Lecture 4: Statistics For Transcriptomics}
\subtitle{BIOINF3005/7160: Transcriptomics Applications}
%\setbeamercovered{transparent} 
\setbeamertemplate{navigation symbols}{} 
\logo{
	\includegraphics[scale=0.3]{figures/UoA_logo_col_vert.png} 
} 
\institute{Bioinformatics Hub, \\The University of Adelaide} 
\date{March 23rd, 2020} 
\subject{BIOINF3005/7160: Transcriptomics Applications} 


\begin{document}

\begin{frame}
\titlepage
\end{frame}

\begin{frame}
\footnotesize
\tableofcontents
\end{frame}



\section{Hypothesis Testing}

\begin{frame}{Hypothesis Testing}

In biological research we often ask:

	\begin{center}
	\textbf{“Is something happening?” or “Is nothing happening?”}
	\end{center}

We might be comparing:

	\begin{itemize}
		\item Cell proliferation in response to antibiotics in media
		\item mRNA abundance in two related cell types
		\item Methylation levels across genomic regions
		\item Allele frequencies in two populations
	\end{itemize}

\end{frame}

\begin{frame}{Hypothesis Testing}

	How do we decide if our experimental results are “significant”?

	\begin{itemize}
		\item Is it normal variability?
		\item What would the data look like if our \textit{experiment had no effect?}
		\item What would our data look like if there was \textit{some kind of effect?}
	\end{itemize}
	
	\textbf{Every experiment is considered as a random sample from all possible repeated experiments.}

\end{frame}

\begin{frame}{Sampling}

Most experiments involve measuring something:

	\begin{itemize}
		\item Discrete values e.g. read counts, number of colonies
		\item Continuous values e.g. Ct values, fluorescence intensity
	\end{itemize}

	\textbf{Every experiment is considered as a random sample from all possible repeated experiments.}

\end{frame}

\begin{frame}{Sampling}

	Many data collections can also be considered as experimental datasets
	
	\begin{block}{Example 1}
	In the 1000 Genomes Project a risk allele for T1D has a frequency of $\pi = 0.07$ in European Populations.
	\end{block}
	
	~\\
	\textbf{Does this mean, the allele occurs in exactly 7\% of Europeans?}

\end{frame}

\begin{frame}{Sampling}

	\begin{block}{Example 2}
	In our in vitro experiment, we found that 90\% of HeLa cells were lysed by exposure to our drug.
	\end{block}
	
		\begin{itemize}
			\item Does this mean that exactly 90\% of HeLa cells will always be destroyed?
			\item What does this say about in vivo responses to the drug?
		\end{itemize}

\end{frame}

\begin{frame}{Population Parameters}

	\begin{itemize}
		\item Experimentally-obtained values represent an \textbf{estimate} of the true effect
		\item More formally referred to as \textit{population-level parameters}
		\item Every experiment is considered a \textit{random sample of the complete population}
		\item Repeated experiments would give a \textbf{different} \textit{(but similar)} estimate
	\end{itemize}

\end{frame}


\end{document}